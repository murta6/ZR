\documentclass[times, utf8, zavrsni]{fer}
\usepackage{booktabs}
\usepackage{indentfirst}

\begin{document}

% TODO: Navedite broj rada.
\thesisnumber{4350}

% TODO: Navedite naslov rada.
\title{Razvoj podatkovnog sloja i aplikacijske logike za potrebe sustava elektroničkog učenja}

% TODO: Navedite vaše ime i prezime.
\author{Alen Murtić}

\maketitle

% Ispis stranice s napomenom o umetanju izvornika rada. Uklonite naredbu \izvornik ako želite izbaciti tu stranicu.
\izvornik

% Dodavanje zahvale ili prazne stranice. Ako ne želite dodati zahvalu, naredbu ostavite radi prazne stranice.
\zahvala{}

\tableofcontents

\chapter{Uvod}
Znanje i prenošenje znanja najznačajniji je faktor sve bržeg razvoja ljudske civilizacije, iako je često nailazilo na prepreke u širenju. Prije početka 20. stoljeća ponajveći je problem bila dostupnost znanja, a ta je prepreka postajala sve manja izumima kao što su tiskarski stroj, sveučilište te obveznim školstvom i masovnom proizvodnjom knjiga. Dostupnog znanja danas je minoran ili čak nepostojeći problem, no to je stvorilo novi izazov - odabir pravog načina učenja. Moderno elektroničko doba samo ga je još pojačalo jer je ogromna količina e-knjiga od korisnika udaljena na samo nekoliko klikova.
\par
Moderni svijet donosi golemu količinu dostupnih informacija što je dovelo do toga da je ljudska pažnja sve neodređenija, a vrijeme fokusa sve kraće. Moderne generacije nemaju imperativ učenja kao njihovi preci, uglavnom zato što nikada nisu iskusile težinu ne-modernog života. Stoga se može dogoditi da čovjek željan znanja jednostavno odustane zbog ogromne količine mogućnosti, a posebno zbog činjenice se samo u ponekim školskim sustavima uči kako učiti.
\par
Shvativši da je znanje i prenošenje znanja izuzetno bitno te da je moderni svijet stvorio nove izazove učenju, postavlja se pitanje kako poboljšati njegovu kvalitetu? Odgovor je relativno jednostavan: pretvoriti učenje u nešto zanimljivo i jednostavno za korištenje, ali ipak izazovno i korisno - specijalizirane aplikacije, tj. sustave za elektroničko učenje.
\par
Ovaj završni rad opisat će što su sustavi za elektroničko učenje, što znači da je takav sustav inteligentan, neke od algoritama inteligencije, kako trebaju izgledati podatkovni i aplikacijski slojevi takvog sustava te objasniti što su oni uopće. U konačnici, temeljem principa navedenih u teoretskom dijelu, opisat će podatkovni i aplikacijski sloj sustava koji sam implementirao.


\chapter{Sustavi za elektroničko učenje}
Ideja sustava za elektroničko učenje je ujediniti postupak prenošenja informacija, upijanja novih činjenica i provjere znanja, baš kao na akademskim institucijama. No, prednosti sustava za elektroničko učenje nad tradicionalnim akademskim institucijama su veća prostorna dostupnost (praktički jedini uvjet sudjelovanja je internetska povezanost) te brže ažuriranje gradiva novim informacijama u odnosu na spori obrazovni sustav.
\par
Sustavi za elektroničko učenje mogu biti dio nekog formalnog obrazovnog sustava ili samostalni. Moodle je primjer sustava za elektroničko učenje čiji je cilj obavljati samo dio obrazovanja, uglavnom provjere znanja i prenošenja informacija (materijala za učenje). Od korisnika se očekuje da sam proučava te materijale ili pohađa predavanja predmeta unesenog na sustav Moodle. Samostalni sustavi za elektroničko učenje trebali bi imati dostupne sve elemente učenja jer samo tako mogu osigurati kvalitetu. Eventualno može nedostajati formalna provjera znanja ako je sustav dizajniran tako da je postupak upijanja novih činjenica rigorozan u korištenju da ga korisnik ni u kojem slučaju ne može preskočiti ili ignorirati. 
\par
Neki od popularnih i besplatnih sustava za elektroničko učenje su Codeacademy, Doulingo i Coursera. Coursera je zanimljiv koncept učenja koji na internet stranicu prenosi predmete stvarnih sveučilišta. Ona je za kranjeg korisnika samostalna, no stvaranje njezinog sadržaja nije. Duolingo Najpoznatiji akamdemski sustav za učenje je Moodle, a na FER-u se također koriste Ahyco i Ferko.
\par
Ahyco, Ferko
\pagebreak
\section{Inteligentni sustavi za elektroničko učenje (ITS)}
Inteligentni sustavi za elektroničko učenje posebna su vrsta sustava za elektroničko učenje koja korisniku pruža reakcije ili upute prilagođene isključivo njemu. Cilj svakog ITS je smanjiti ovisnost korisnika sustava o ljudskim učiteljima. Za razliku od ljudi, elektronički sustavi se ne umaraju, ne stare, gotovo uvijek imaju vremena za učenika. Naravno, inteligentni sustavi za elektroničko učenje također imaju svoje limite: kvalitetni su koliko je kvalitetan dizajner sustava, ne mogu pratiti fizičke reakcije korisnika i iskustveno zaključivati iz njih te, ipak, ne mogu biti potpuno individualni jer bi u tom slučaju količina praćenih podataka morala biti enormna. 

\subsection{Povijest ITS-a i zaključci iz nje}

\subsubsection{Mehanički strojevi za učenje}
Ideja inteligentnih sustava za učenje nije nastala tek u 21. stoljeću, dapače, već u 17. st. dvojica od najvećih europskih matematičara, Blaise Pascal i Gottfried Wilhelm Leibniz razmatrali su koncepte zaključivanja uz pomoć strojeva. Pravi zamah elektroničko učenje dobiva u 20.
\par TODO nekakva slika 
\par
Prvi značajan korak prema učenju uz pomoć strojeva je napravio Sidney Leavitt Presley, profesor na Sveučilištu Ohio, koji je izumio stroj za samoocjenjivanje. Korisnik je odgovarao na pitanja s višestrukim izborom, stroj bi spremao odgovore te na kraju seta pitanja dao povratnu poruku o točnosti odgovora. Taj pristup koristi se i danas u većini akademskih sustava za ocjenivanje, sva tri, Moodle, Ferko i Ahyco ga koriste.
\par
Iako su mehanički strojevi pružili kvalitetan početak ideje, jasno je da zbog neefleksibilnosti i skupoće izrade nikada nisu mogli prijeći fazu demonstracije. No, računalne aplikacije je lako umnožavati i mijenjati. Zato su one pogodne za razvoj sustava za učenje, ovog puta elektroničkih.

\subsubsection{Početak elektroničkih sustava za učenje}
U ranim godinama razvoja računala i računarstva napravljeni su mnogi koncepti bliskih tematika, npr. Turingov test, ideja umjetne inteligencije i strojnog učenja, no sustavi za učenje kao koncept kakav danas postoji je relativno zanemaren. Jedini važan napredak su definiranja gramatika u računarstvu, kao jednog od budućih načina parsiranja korisnikovih odgovora. No, dolaskom računala u širu upotrebu, u američkim školama se stvaraju CAI (\textit{Computer-Assisted instruction}) projekti s ciljem učenja programiranja. CAI sustav je takav da se korisniku prezentira neki materijal s uputom korištenja, nakon čega on na \textit{input-output} principu komunicira s računalom. Najbolji primjer nečega sličnog CAI sustavu u današnje vrijeme je Codeacademy. U početku se CAI koristio u učionicama, a ne individualno, ali na sličnom principu korištenja.
\par TODO NEKA SLIKA ZA CAI
\par
Velik korak za stvaranje ITS-a napravio je Jaime Carbonell koji je 1970. postavio tezu da "sustavi za učenje ne moraju biti samo alat, nego i učitelj". U 1970.-ima je najpopularnije novo područje bila umjetna inteligencija temeljena na znanju. Najpoznatiji primjer takvog sustava je \textit{Dendral}, sustav zaključivanja o kemijskim strukturama korištenim u organskoj kemiji. Izgradnja složenijih sustava je bila lakša zbog poboljšanja brzine računala što je fokus rada konačno maknulo s tehnologije na tehniku. 
\par
Početkom 1980. umjetna inteligencija se pomiče k neuronskim mrežama i kognitivnoj psihologiji što otvara prostor razvoju sustava za elektroničko učenje zbog sličnih područja interesa. To dovodi do stvaranja koncepta ICAI-ja (\textit{Intelligent Computer Assisted Instruction}), vrste CAI sustava koji bi radio na principu prilagođenog posluživanja instrukcija. ICAI je jako blizu modernom konceptu ITS-a, jedina razlika između njih je što ICAI ne pokušava modelirati znanje korisnika.

\subsubsection{ITS u doba interneta i analize podataka}
Najvažniji korak za veću primjenu elektroničkog učenja je početak masovnog korištenja interneta. Internet je konačno omogućio neovisnost fizičkih pozicija korisnika sustava i kvalitete znanja koju mu sustav može pružiti. Slabost prijašnjih rješenja je bila u neadekvatnoj aktualizaciji sustava zbog činjenice da su se morali distribuirati na prijenosnim medijima. Nije slučajno da su svi sustavi koje sam na početku nabrojao internetske stranice.
\par  TODO nekakva slika internet connecting world
\par
Analiza podataka, \textit{data mining}, \textit{big data} i \textit{data science} je najveći korak u razvoju inteligencije ITS-a. Iako su principi analize podataka bili matematički postavljeni i prije 1990.-ih, rastom brzine računala, popularizacijom SQL-a i internet tražilica, analiza podataka doživjela je pravi \textit{boom} zadnjih 20-tak godina. Razlog tolike važnosti analize podataka u inteligenciji ITS-a je što kompleksnost procesa učenja (čime se bavi kognitivna psihologija), "udaljnost" sustava i potreba za prilagođenim (personaliziranim) sadržajem korisniku. Ti faktori se jednostavno ne mogu opisati malim brojem varijabli. Svaki učitelj kroz godine upoznaje nove učenike te nadopunjuje svoje znanje i tehniku, baš kao što bi ITS bio spreman za nadogradnju temeljem iskustva korištenja.

\subsection{Komponente ITS-a}
Većina modernih ITS-ova dijeli se na 4 komponente rada: domenski i korisnički model, model učenja te korisničko sučelje. Tri prvonavedene komponente se nazivaju modeli zato što su pojednostavljena reprezentacija stvarnih pojava, koncepata i sl. One nemaju nužno veze s modelom kao aplikacijskim slojem, iako se uglavnom oslanjaju na njega.

\subsubsection{Domenski model}
Domenski model je dio sustava čiji je smisao organizirati znanje u jedinice spremne za pohranu u bazu podataka. Svaki sustav osmišljava vlastitu podjelu znanja s obzirom na različitu širinu podjele i broj razina. U pravilu je neizbježna podjela na predmete u složenim sustavima i koncepte unutar predmeta, a na autorima i administatorima sustava je definirati više ili niže razine u odnosu na koncepte. Granulacija znanja je zato prilično kompliciran posao većoj količini autora, no budući da je kreiranje domenskog modela kreativan posao, u procesu zamišljanja treba sudjelovati što veći broj ljudi.
\par
Domenski model ne uključuje samo podjelu znanja, nego i pravila te strategije koje treba naučiti. Oni se mogu prikazati eksplicitno formulama i materijalima za učenje ili implicitno redoslijedom prikaza pitanja ili slično. Implicitno zadavanje pravila i strategija tipično je za formalne školske sustave zato što postoji određen redoslijed predavanja te nastavnici očekuju da učenici/studenti poznaju gradiva prethodnih predavanja. U sustavima za elektroničko učenje takav je pristup teže izvesti te se pravila i strategije najčešće zadaju eksplicitno u obliku najniže razine podjele u bazi podataka, iako korisnik uglavnom tu eksplicitnost ne vidi.
\par
Prilikom kreiranja domenskog modela važno je misliti na to da je takav model kvalitetna reprezentacija različitih vrsta znanja, da bude efikasan i razumljiv za kreiranje algoritama te da može odgovoriti na potrebe korisničkog modela.
\par
Tipični domenski model je kvalitetno organizirana baza podataka te pravilno uneseni podaci. Za sve iznad toga se brinu ostale komponente, počevši sa korisničkim modelom.

\subsubsection{Korisnički model}
Učenički, studentski ili jednostavno korisnički model je komponenta sustava čija se svrha većim dijelom preklapa s domenskom komponentom i modelom učenja. Razlog odvajanja korisničkog modela je bolja analiza znanja pojedinog korisnika. 
\par
Korisnički model su uglavnom algoritmi koji analiziraju točnost odgovora (te možda daju dodatne savjete korisniku) i algoritmi koji omogućuju prikaz analize odgovora korak po korak. Zbog relativno male opsežnosti, neki izvori zanemaruju korisnički model cijepajući njegove dijelove u domenski i model učenja.

\subsubsection{Model učenja}
Nakon postavljanja domenskog i korisničkog modela, ostatak aplikacijske logike sadržan je u modelu učenja. Upravo je model učenja najbitniji dio inteligentnog sustava za elektroničko učenje jer on zamjenjuje ljudskog učitelja. Ovoj komponenti pripadaju algoritmi koji na temelju rezultata prethodno navedenih evaluiraju znanje korisnika, algoritmi korisnikove navigacije sustavom te algoritmi posluživanja pitanja.
\par
Model učenja srž je ITS-a te mjesto 

\subsubsection{Korisničko sučelje}


\subsection{Karakteristike dobrih sustava}
Nakon definiranja komponenata sustava za inteligentno učenje, potrebno je razmotriti ka

\subsubsection{Parametrizirana pitanja} parametrizirana pitanja

\subsubsection{Prilagođene reakcije}

\subsubsection{Praćenje tipkanja}
umjesto praćenja fizičkih reakcija

\subsubsection{Sigurnost}

\subsubsection{Kvalitetni algoritmi}

\subsubsection{Principi dizajna}

\subsection{Algotimi sustava za inteligentno učenje}

\subsection{Poznati sustavi}
Coursera? Doulingo? Računalne igrice (zbog AI, Dishonored)? Urban Jungle

\chapter{ TODO NASLOV: slojevi}
Sve kvalitetno oblikovane moderne aplikacije moraju sadržavati nekoliko slojeva rada. Slojevita arhitektura omogućuje sigurno zadržavanje dva najvažnija logička načela dobrog oblikovanja: nadogradnju bez promjene i načelo jedinstvene ovisnosti. Takav pristup također intervencije u kod aplikacije čini lakšim. Arhitektura koju koristim je MVC (Model - View - Controller). Prevedeno na hrvatski, model je podatkovni sloj, controller je aplikacijska logika, a view pogled. Najveća prednost ovakvog pristupa je jednostavna i intuitivna višeplatformska modularnost - podatkovni sloj je zajednički za sve platforme, aplikacijsku logiku može koristiti više različitih platformi (npr. kod pisan u Javi za Android i web aplikacije ili C\# za Windows 10 i web), a svaka platforma određuje svoje korisničko sučelje, tj. pogled.
\\TODO NEKAKVA SLIKA ŠTAJAZNAM
\par
Podatkovni sloj je sloj pohrane, dohvata i ažuriranja podataka, gotovo uvijek izveden u obliku baze podataka te ponekad pripadnih tehnologija kao npr. Entity Framework koji mapira objekte i bazu podataka. Podaci spremljeni u podatkovnom sloju moraju biti u \textit{display-neutral} obliku, što znači da bez daljnje obrade nisu previše pogodni za sami prikaz. Cilj takvog pristupa spremanja podataka je imati bazu u prve tri normalne forme te posljedično očuvati performanse baze na visokoj razini čak i uz veću količinu podataka spremljenih u nju. Podatkovni sloj implementiran u ovom završnom radu je SQL baza podataka te Entity Framework.
\par
Aplikacijska logika je sloj upravljanja korisničkim zahtjevima, a radi kao poveznica između prikaza i podatkovnog sloja. Aplikacijska logika prima naredbe koje je korisnik zadao na pogledu, zatim provodi neke operacije nad modelom, uzima podatke iz modela te ih u \textit{display-neutral} obliku šalje pogledu. Ona je uglavnom implementirana kao nekoliko slojeva različitih poslova, npr. \textit{utlity} sloj rada s bazom, sloj koji prima podatke od pogleda te algoritamski sloj koji provodi operacije. U ovom završnom radu aplikacijska logika će upravo raditi na taj način: statičke klase rada s bazom, klase algoritama odabira pitanja te klase koje primaju pozive operacija s korisničkog sučelja.
\par
Pogled, tj. korisničko sučelje je način na koji se aplikacija prezentira kranjem korisniku te način na koji je on koristi. Njegov je cilj rada obraditi podatke tako da ih može lijepo i kvalitetno prikazati korisniku te slati zahtjeve korisnika aplikacijskoj logici. Budući da zadatak ovog ZR nije implementirati taj sloj aplikacije, jedino korisničko sučelje koje će se koristiti je ono za administratora koji mora moći dodavati nova pitanja, generirati testove i slično.
\par
Uspoređujući slojeve inteligentnog sustava za elektroničko učenje i MVC višeslojne aplikacije, vidimo jasnu povezanost izemđu domenskog modela i modela kao dio MVC-a, controllera te korisničkog modela i modela znanja te grafičkog sučelja i pogleda. Upravo to je smisao MVC pristupa, jednostavno je logičan za ogromnu količinu situacija. Zašto onda ovaj završni rad navodi oba? Zato što je MVC tehnološki presjek, a slojevi ITS-a logički. Njihovo poklapanje je potvrda da je aplikacija projektirana na ispravan način. Filozofskiji odgovor bi bio da je MVC pogled programskog inženjerstva na aplikaciju, a slojevi ITS-a računarske znanosti.

\chapter{Opis implementiranog rješenja}
Opis rješenja

\section{Korištene tehnologije}
SQL, C\#, Java, Entity Framework,
SQL Management Studio, Visual Studio, Eclipse, Android Studio

\section{ER dijagram}
ER

\section{Mogućnosti sustava}
mogućnosti

\subsection{Parametrizirana pitanja}

\subsection{baza stanja, stablo znanja, whatever}

\subsection{Mogućnosti administratora}

\subsection{Mogućnosti korisnika}

\section{Logika posluživanja pitanja}
logika

\chapter{Zaključak}
Zaključak.

\bibliography{literatura}
\bibliographystyle{fer}

\begin{sazetak}
Sažetak na hrvatskom jeziku.

\kljucnerijeci{Ključne riječi, odvojene zarezima.}
\end{sazetak}

% TODO: Navedite naslov na engleskom jeziku.
\engtitle{Application logic and data layer development for an e-learning system}
\begin{abstract}
Abstract.

\keywords{Keywords.}
\end{abstract}

\end{document}
